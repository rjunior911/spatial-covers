 %% \usepackage{natbib}
%\usepackage[sorting=none,citestyle=authoryear]{biblatex}
\usepackage{geometry} %set the font and margins
\usepackage{bibentry} %allows for inline citations before the bibliography (say in a footnote or resume)
\usepackage[final]{graphicx} % Allows including images
\usepackage{amssymb}
\usepackage{amsmath}
\usepackage{amsthm} 
\usepackage{tabularx}
\usepackage{tikz} 
\usepackage{tikz-cd} %cd for Commutative Diagrams
% \usepackage{xypic}
\usepackage{caption}
\usepackage{subcaption}
\usepackage{wrapfig}
%% \usetikzlibrary{babel}

%had to redefine sections and subsections to put line breaks
% \usepackage[parfill]{parskip}    

% \makeatletter
% \def\section{\@startsection{section}{3}%
%   \z@{.5\linespacing\@plus.7\linespacing}{.1\linespacing}%
%   {\normalfont\itshape}}
% \def\subsection{\@startsection{subsection}{3}%
%   \z@{.5\linespacing\@plus.7\linespacing}{.1\linespacing}%
%   {\normalfont\itshape}}
% \def\subsubsection{\@startsection{subsubsection}{3}%
%   \z@{.5\linespacing\@plus.7\linespacing}{.1\linespacing}%
%   {\normalfont\itshape}}
% \makeatother

% ========================================
%TAKE 17 on lowercase math script fonts
% ========================================
% \let\mathbbalt\mathbb
% \usepackage{fontspec}
% \usepackage{unicode-math}
\usepackage[utf8]{inputenc}
% \usepackage{mathalfa} % more font options
\usepackage[cal=esstix,
% bb=ams,
% frak=ams,
scr=rsfs
]{mathalpha}
% \usepackage[cal=boondoxo]{mathalfa}
\usepackage{mathabx}

\usepackage{amsfonts} % more font options
% \usepackage{mathrsfs} %doesn't work, maybe need to fix texlive?
% \usepackage[mathscr]{eucal}
% \usepackage{boondox-cal}
% \usepackage{calligra}%breaks on Mac
% \DeclareMathAlphabet{\mathcal}{T1}{calligra}{m}{n}
% ========================================
% \let\mathbb\mathbbalt% UNIVERSAL RESET TO ORIGINAL \mathbb

\usepackage{pdfsync}

% \usepackage{ntheorem}
\usepackage{booktabs} % Allows the use of \toprule, \midrule and \bottomrule in tables

\newcolumntype{L}{>{\centering\arraybackslash}p{2cm}}
\newcolumntype{M}{>{\centering\arraybackslash}p{0.45t\textwidth}}
\newcolumntype{E}{>{\centering\arraybackslash}p{0.1\textwidth}}
\newcommand{\de}[1]{\textbf{#1}}

\newtheorem{thm}{Theorem}
\newtheorem{lem}{Lemma}
\newtheorem{conj}{Conjecture}
\newtheorem{conjecture}{Conjecture}
\newtheorem{prop}{Proposition}
\newtheorem*{definition}{Definition}
\newtheorem*{principle}{Principle}
\newtheorem*{project}{Project}
\newtheorem*{moral}{Moral}
\newtheorem*{question}{Question}
\newtheorem*{intuition}{Intuition}
\newtheorem*{fact}{Fact}
\newtheorem*{xample}{Example}

\renewcommand\qedsymbol{$\blacksquare$}

%Below lies the necessary commands to copy nlab diagrams into my tex code
\newcommand{\itexarray}[1]{\begin{matrix}#1\end{matrix}}

% math-mode versions of \rlap, etc
% from Alexander Perlis, "A complement to \smash, \llap, and lap"
%   http://math.arizona.edu/~aprl/publications/mathclap/
\def\clap#1{\hbox to 0pt{\hss#1\hss}}
\def\mathllap{\mathpalette\mathllapinternal}
\def\mathrlap{\mathpalette\mathrlapinternal}
\def\mathclap{\mathpalette\mathclapinternal}
\def\mathllapinternal#1#2{\llap{$\mathsurround=0pt#1{#2}$}}
\def\mathrlapinternal#1#2{\rlap{$\mathsurround=0pt#1{#2}$}}
\def\mathclapinternal#1#2{\clap{$\mathsurround=0pt#1{#2}$}}

\newcommand{\darr}{\downarrow}

